% In node, use accepting for initial states.
% In node, use fill=orange!50 for final states.
% In edge, use \cguard{} for guards.
% In edge, use \creset{} for assignments.

\begin{figure}[h!]
\newcommand{\cguard}{\textcolor{blue}}
\newcommand{\creset}{\textcolor{green!50!black}}

\begin{center}
\begin{tikzpicture}[
	->,
	>=stealth',
	shorten >=2pt, 
	auto,
	scale=0.8,
    transform shape, 
    align=center,
    state/.style={thick, circle, draw}
    ] 
    
	\node[state] (s0) {s0};
	\node[state, right = 2cm of s0] (s1) {s1}; 
	\node[state, right = 4cm of s1] (s2) {s2}; 
	\node[state, right = 2cm of s2] (s3) {s3};
	\node[state, right = 2cm of s3] (s4) {s4};
	\node[state, right = 4cm of s4] (s5) {s5};   
		    
	\draw [line width=0.35mm]
	(s0) edge node{\cguard{$y == p$} \\ \creset{$x := 0$}}(s1)
	(s1) edge node{\cguard{$x == p \; \&\& \; y > a$}} (s2)
	(s2) edge node{\creset{$x := 0,$} \\ \creset{$y := 0$}} (s3)
	(s3) edge node{\cguard{$y == p$} \\ \creset{$x := 0$}}(s4)
	(s4) edge node{\cguard{$x == p \; \&\& \; y < b$}} (s5)
    ;
\end{tikzpicture}
\end{center}

\caption{Tile 3. Forced interval: $p \in (\frac{a}{2}, \frac{b}{2})$}
\label{tile 3}
\end{figure}

